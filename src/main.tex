\documentclass{article}

\title{Solar System}
\author{Abhishek }
\date{July 2024}

\begin{document}

\maketitle

\section{Introduction}
Introduction to solar system.
\section{The Sun}
The Sun is the star at the center of the Solar System. It is a nearly perfect sphere of hot plasma and is the most important source of energy for life on Earth.

\section{Planets}
\subsection{Mercury}
Mercury is the smallest planet in the Solar System and the closest to the Sun. It has no natural satellites.

\subsection{Venus}
Venus is the second planet from the Sun. It is similar in structure and size to Earth, but it has a thick, toxic atmosphere that traps heat.

\subsection{Earth}
Earth is the third planet from the Sun and the only astronomical object known to harbor life. It has one natural satellite, the Moon.

\subsection{Mars}
Mars is the fourth planet from the Sun. It is known as the Red Planet due to its reddish appearance, which is caused by iron oxide on its surface.

\subsection{Jupiter}
Jupiter is the fifth planet from the Sun and the largest in the Solar System. It has a strong magnetic field and at least 79 moons, including the four large Galilean moons.

\subsection{Saturn}
Saturn is the sixth planet from the Sun, known for its extensive ring system. It has at least 83 moons, with Titan being the largest.

\subsection{Uranus}
Uranus is the seventh planet from the Sun. It has a unique tilt that causes it to rotate on its side, and it has 27 known moons.

\subsection{Neptune}
Neptune is the eighth and farthest known planet from the Sun in the Solar System. It has 14 known moons, with Triton being the largest.

\section{Dwarf Planets}
\subsection{Pluto}
Pluto is the most famous dwarf planet in the Solar System. It was reclassified from a planet to a dwarf planet in 2006.

\subsection{Eris}
Eris is one of the largest known dwarf planets in the Solar System. It is located in the scattered disk, a distant region of the Solar System.

\section{Other Celestial Objects}
\subsection{Asteroids}
Asteroids are small rocky bodies that orbit the Sun, mostly found in the asteroid belt between Mars and Jupiter.

\subsection{Comets}
Comets are icy bodies that release gas or dust. They are often characterized by their bright comas and long tails.

\subsection{Meteoroids}
Meteoroids are small particles from comets or asteroids that are in space. When they enter Earth's atmosphere, they become meteors.

\section{Conclusion}
The Solar System is a vast and complex system that continues to fascinate scientists and astronomers. Understanding its components helps us learn more about the universe and our place in it
\end{document}
